\documentclass[UTF8]{ctexart}
\usepackage{geometry, CJKutf8}
\geometry{margin=1.5cm, vmargin={0pt,1cm}}
\setlength{\topmargin}{-1cm}
\setlength{\paperheight}{29.7cm}
\setlength{\textheight}{25.3cm}

% useful packages.
\usepackage{amsfonts}
\usepackage{amsmath}
\usepackage{amssymb}
\usepackage{amsthm}
\usepackage{enumerate}
\usepackage{graphicx}
\usepackage{multicol}
\usepackage{fancyhdr}
\usepackage{layout}
\usepackage{listings}
\usepackage{float, caption}

\lstset{
    basicstyle=\ttfamily, basewidth=0.5em
}

% some common command
\newcommand{\dif}{\mathrm{d}}
\newcommand{\avg}[1]{\left\langle #1 \right\rangle}
\newcommand{\difFrac}[2]{\frac{\dif #1}{\dif #2}}
\newcommand{\pdfFrac}[2]{\frac{\partial #1}{\partial #2}}
\newcommand{\OFL}{\mathrm{OFL}}
\newcommand{\UFL}{\mathrm{UFL}}
\newcommand{\fl}{\mathrm{fl}}
\newcommand{\op}{\odot}
\newcommand{\Eabs}{E_{\mathrm{abs}}}
\newcommand{\Erel}{E_{\mathrm{rel}}}

\begin{document}

\pagestyle{fancy}
\fancyhead{}
\lhead{姚杭希, 3230102918}
\chead{数据结构与算法第五次作业}
\rhead{Oct.30th, 2024}

\section{测试程序的设计思路}

下面我将简述 ALV Tree 中的 remove 的设计思路,与之前BST中的 remove 其实非常类似 

首先我们创建了一个 detachmin 函数,返回以 t 为根节点树中的最小值,通过递归实现,
只要当前节点还存在左儿子,那么就往左子树走直到走到叶子节点为止。

其次我将简述 remove 函数中的大致思路:
首先我们找到 x 节点的所在位置,若 x 左右儿子都存在,那么通过这个函数返回的右子树最小节点将代替被删除节点。
否则则直接返回左右儿子中有的那一边同时将原本 t 节点的指针转移到替换节点上去。
最后使用了 balance 函数来帮助我们实现平衡

在 balance 函数中我们具体考虑四种可能会出现的情况,由于对称性我们只需考虑两种,
分别是左左和左右,右右和右左只需对称编写代码即可
前一种情况可以通过对不平衡节点进行一次右旋转来修复。
而对于后一种情况首先对不平衡节点的左子节点进行左旋转,然后对不平衡节点本身进行右旋转。

我用 valgrind 进行测试,发现没有发生内存泄露。


\end{document}

%%% Local Variables: 
%%% mode: latex
%%% TeX-master: t
%%% End: 
