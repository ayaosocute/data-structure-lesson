\documentclass[UTF8]{ctexart}
\usepackage{geometry, CJKutf8}
\geometry{margin=1.5cm, vmargin={0pt,1cm}}
\setlength{\topmargin}{-1cm}
\setlength{\paperheight}{29.7cm}
\setlength{\textheight}{25.3cm}

% useful packages.
\usepackage{amsfonts}
\usepackage{amsmath}
\usepackage{amssymb}
\usepackage{amsthm}
\usepackage{enumerate}
\usepackage{graphicx}
\usepackage{multicol}
\usepackage{fancyhdr}
\usepackage{layout}
\usepackage{listings}
\usepackage{float, caption}

\lstset{
    basicstyle=\ttfamily, basewidth=0.5em
}

% some common command
\newcommand{\dif}{\mathrm{d}}
\newcommand{\avg}[1]{\left\langle #1 \right\rangle}
\newcommand{\difFrac}[2]{\frac{\dif #1}{\dif #2}}
\newcommand{\pdfFrac}[2]{\frac{\partial #1}{\partial #2}}
\newcommand{\OFL}{\mathrm{OFL}}
\newcommand{\UFL}{\mathrm{UFL}}
\newcommand{\fl}{\mathrm{fl}}
\newcommand{\op}{\odot}
\newcommand{\Eabs}{E_{\mathrm{abs}}}
\newcommand{\Erel}{E_{\mathrm{rel}}}

\begin{document}

\pagestyle{fancy}
\fancyhead{}
\lhead{姚杭希, 3230102918}
\chead{数据结构与算法第五次作业}
\rhead{Oct.30th, 2024}

\section{测试程序的设计思路}

下面我将简述 remove 的设计思路:
首先我们创建了一个 detachmin 函数辅助我们进行操作,
它的作用是返回以 t 为根节点树中的最小值,具体通过递归实现,
只要当前节点还存在左儿子,那么就往左子树走直到走到叶子节点为止。

其次我将简述 remove 函数的大致思路:
首先我们找到 x 节点的所在位置,若 x 左右儿子都存在,
那么通过这个函数返回的右子树最小节点将代替被删除节点。
否则则直接返回左右儿子中有的那一边。
同时将原本 t 节点的指针转移到替换节点上去。

接下来我对程序进行了测试:
我首先构建了一棵满二叉树,并逐步测试了各种情况下的 remove 
各种不同的情况分别写在主函数的注释中了:删除叶子节点,只有左儿子,只有右儿子,左右儿子都有等情况。
测试结果如下:

Initial Tree:
1
3
4
5
6
7
8
10
12
15
18
Tree after removing 8:
1
3
4
5
6
7
10
12
15
18
Tree after removing 5:
1
3
4
6
7
10
12
15
18
Tree after removing 7:
1
3
4
6
10
12
15
18
Tree after removing 12 and 15:
1
3
4
6
10
18
Tree after making empty:
Empty tree
测试结果一切正常。

我用 valgrind 进行测试,发现没有发生内存泄露。


\end{document}

%%% Local Variables: 
%%% mode: latex
%%% TeX-master: t
%%% End: 
