\documentclass[UTF8]{ctexart}
\usepackage{geometry, CJKutf8}
\geometry{margin=1.5cm, vmargin={0pt,1cm}}
\setlength{\topmargin}{-1cm}
\setlength{\paperheight}{29.7cm}
\setlength{\textheight}{25.3cm}

% useful packages.
\usepackage{amsfonts}
\usepackage{amsmath}
\usepackage{amssymb}
\usepackage{amsthm}
\usepackage{enumerate}
\usepackage{graphicx}
\usepackage{multicol}
\usepackage{fancyhdr}
\usepackage{layout}
\usepackage{listings}
\usepackage{float, caption}

\lstset{
    basicstyle=\ttfamily, basewidth=0.5em
}

% some common command
\newcommand{\dif}{\mathrm{d}}
\newcommand{\avg}[1]{\left\langle #1 \right\rangle}
\newcommand{\difFrac}[2]{\frac{\dif #1}{\dif #2}}
\newcommand{\pdfFrac}[2]{\frac{\partial #1}{\partial #2}}
\newcommand{\OFL}{\mathrm{OFL}}
\newcommand{\UFL}{\mathrm{UFL}}
\newcommand{\fl}{\mathrm{fl}}
\newcommand{\op}{\odot}
\newcommand{\Eabs}{E_{\mathrm{abs}}}
\newcommand{\Erel}{E_{\mathrm{rel}}}

\begin{document}

\pagestyle{fancy}
\fancyhead{}
\lhead{姚杭希, 3230102918}
\chead{四则运算作业}
\rhead{12.14th, 2024}

\section{程序的设计思路}

整个expression_evaluator程序的主要思路是将用户输入的中缀表达式
转换为后缀表达式,然后计算后缀表达式的结果

中缀到后缀的转换:
中缀表达式转换为后缀表达式的过程主要通过使用栈来处理运算符的优先级和括号。转换规则如下:
遇到数字直接输出到后缀表达式
遇到运算符,从栈中弹出所有优先级大于或等于当前运算符的运算符,然后将当前运算符压入栈
遇到左括号直接压入栈中
遇到右括号则依次弹出栈顶运算符并输出,直到遇到左括号为止,左括号只弹出不输出
表达式扫描完毕后,将栈中剩余的运算符依次弹出并输出

后缀表达式的计算:
后缀表达式的计算规则如下:
从左到右扫描后缀表达式
遇到操作数时,将其压入栈中
遇到运算符时,从栈中弹出顶部的两个元素,先弹出的作为右操作数,后弹出的作为左操作数,将运算结果再压入栈中
表达式扫描完毕后,栈顶元素即为最终结果

特殊处理:
科学计数法的处理:
在扫描和转换过程中,加入了对e的判断使其可以正确处理科学计数法

在该程序中会在以下几种情况下输出ILLEGAL:

非法字符:如果输入的表达式中包含不被识别的字符,如任何非数字,非运算符,非括号的字符
括号不匹配:如果在表达式中左右括号的数量不匹配,或者括号的使用顺序不正确
运算符使用错误:例如果表达式以二元运算符开始或结束或有连续的二元运算符
数字格式错误:如果数字的格式不正确,如多个小数点或科学计数法格式错误

\section{测试的结果}

我们给出了各种不同的样例来测试不同情况下程序的输出正确性
Enter an expression: (2e2+3)+4/2
Result: 205
Enter an expression: (2+3)+2)+1
ILLEGAL
Enter an expression: 2+3++1
ILLEGAL
Enter an expression: ((1+2)/(2-1))
Result: 3
以上几种较为复杂的样例程序均未出错

\end{document}

%%% Local Variables: 
%%% mode: latex
%%% TeX-master: t
%%% End: 
